%% 
% * <jhonatanjaam@gmail.com> 2018-01-29T11:12:59.680Z:
%
% ^.
% * <jhonatanjaam@gmail.com> 2018-01-29T11:12:57.858Z:
%
% ^.
%% Copyright 2007, 2008, 2009 Elsevier Ltd
%% 
%% This file is part of the 'Elsarticle Bundle'.
%% ---------------------------------------------
%% 
%% It may be distributed under the conditions of the LaTeX Project Public
%% License, either version 1.2 of this license or (at your option) any
%% later version.  The latest version of this license is in
%%    http://www.latex-project.org/lppl.txt %% and version 1.2 or later is part of all distributions of LaTeX
%% version 1999/12/01 or later.
%% 
%% The list of all files belonging to the 'Elsarticle Bundle' is
%% given in the file `manifest.txt'.
%% 
%% Template article for Elsevier's document class `elsarticle'
%% with harvard style bibliographic references
%% SP 2008/03/01

\documentclass[preprint,12pt,authoryear]{elsarticle}
%\documentclass[final,5p,times,authoryear]{elsarticle}
% \documentclass[final,1p,times,twocolumn,authoryear]{elsarticle}

%% Use the option review to obtain double line spacing
%% \documentclass[authoryear,preprint,review,12pt]{elsarticle}

%% Use the options 1p,twocolumn; 3p; 3p,twocolumn; 5p; or 5p,twocolumn
%% for a journal layout:
%% \documentclass[final,1p,times,authoryear]{elsarticle}
% \documentclass[final,1p,times,twocolumn,authoryear]{elsarticle}
%% \documentclass[final,3p,times,authoryear]{elsarticle}
%% \documentclass[final,3p,times,twocolumn,authoryear]{elsarticle}
%% \documentclass[final,5p,times,authoryear]{elsarticle}
%% \documentclass[final,5p,times,twocolumn,authoryear]{elsarticle}

%% For including figures, graphicx.sty has been loaded in
%% elsarticle.cls. If you prefer to use the old commands
%% please give \usepackage{epsfig}

\usepackage{graphicx}
\usepackage{subfigure}
\usepackage{subfigmat}
\usepackage{multicol}
\usepackage[table]{xcolor} 
%\usepackage{cite}
\usepackage{amsmath}
\usepackage{float}
\usepackage[T1]{fontenc} % Use 8-bit encoding that has 256 glyphs
\usepackage[utf8x]{inputenc}
%% The amssymb package provides various useful mathematical symbols
\usepackage{amsmath,amssymb} 
\usepackage{hyperref}
\usepackage{setspace} 
%to use mathlarger
\usepackage{relsize}
%% The lineno packages adds line numbers. Start line numbering with
%% \begin{linenumbers}, end it with \end{linenumbers}. Or switch it on
%% for the whole article with \linenumbers after \end{frontmatter}.
\usepackage{lineno}
\usepackage{upquote}
\doublespacing
\usepackage[vmargin=1in]{geometry}
\usepackage{tgtermes}
%\usepackage{xfrac,fontspec,unicode-math}

%%% FFF FFF FFF FFF FFF FFF FFF FFF FFF FFF
\usepackage{soul}
%%% FFF FFF FFF FFF FFF FFF FFF FFF FFF FFF

%% The amsthm package provides extended theorem environments
%% \usepackage{amsthm}

%% The lineno packages adds line numbers. Start line numbering with
%% \begin{linenumbers}, end it with \end{linenumbers}. Or switch it on
%% for the whole article with \linenumbers.
%% \usepackage{lineno}

\journal{Computers $\&$ Fluids}

\begin{document}

%\setmathfont{Cambria_Math}
%\setmathfont[version=termes]{TeX Gyre Termes Math}
%\setmathfont{TeX Gyre Termes Math}
%\setmathfont{Cambria Math}
%\setmathfont{Cambria Math}

\begin{frontmatter}

%% Title, authors and addresses

%% use the tnoteref command within \title for footnotes;
%% use the tnotetext command for theassociated footnote;
%% use the fnref command within \author or \address for footnotes;
%% use the fntext command for theassociated footnote;
%% use the corref command within \author for corresponding author footnotes;
%% use the cortext command for theassociated footnote;
%% use the ead command for the email address,
%% and the form \ead[url] for the home page:
% \tnotetext[label1]{}
%% \author{Name\corref{cor1}\fnref{label2}}
%% \ead{email address}
%% \ead[url]{home page}
%% \fntext[label2]{}
%% \cortext[cor1]{}
%% \address{Address\fnref{label3}}
%% \fntext[label3]{}

\title{Direct Numeric Simulation of Diffusion flame}


%% use optional labels to link authors explicitly to addresses:
%% \author[label1,label2]{}
%% \address[label1]{}
%% \address[label2]{}

\author{Jhonatan Andres Aguirre Manco\corref{cor1}}
%\ead{jhonatan@lcp.inpe.br}
%\ead[url]{http://lattes.cnpq.br/6293462403216554}
%\address{Laboratório  Associado de Combustão e propulsão, 
%Instituto Nacional de Pesquisas Espaciais, 
%Cachoeira Paulista, SP-Brasil, 12630-000}
\cortext[cor1]{Principal corresponding author}

\author{Mathues Castro\corref{cor1}}
\author{Cesar Cristaldo\corref{cor1}}
\author{Marcio Teixeira de Mendonca\corref{cor2}}
\author{Fernando Fachini Filho\corref{cor1}}
%\ead{jhonatan@lcp.inpe.br}%\ead[url]{http://lattes.cnpq.br/6293462403216554}%\address{Laboratório  Associado de Combustão e propulsão, %Instituto Nacional de Pesquisas Espaciais, %Cachoeira Paulista, SP-Brasil, 12630-000}
%\cortext[cor1]{Principal corresponding author}

%\ead{marciomtm@fab.mil.br}
%\ead[url]{http://lattes.cnpq.br/9358950208993789}
%\cortext[cor2]{Corresponding author}
%\address{Instituto de Aeronáutica e Espaço, 
%Departamento de Ciência e Tecnologia Aeroespacial, 
%São José dos Campos, SP-Brasil,12228-904}

%\author{Phillips John Morris\corref{cor2}}
%\ead{pjm@psu.edu}
%\ead[url]{http://www2.aero.psu.edu/morris.html}
%\address{The Pennsylvania State University, 233C Hammond Building, University Park, PA 16802, USA}
%
\begin{abstract}
\end{abstract}

\begin{keyword}
%% keywords here, in the form: keyword \sep keyword
%% PACS codes here, in the form: \PACS code \sep code
%% MSC codes here, in the form: \MSC code \sep code
%% or \MSC[2008] code \sep code (2000 is the default)
Mixing Layer  
\sep Combustion 
\sep Diffusion Flame 
\sep Hydrodynamic stability
\end{keyword}

\end{frontmatter}

\linenumbers
%% main text
% \section{}
% \label{}
%% The Appendices part is started with the command \appendix;
%% appendix sections are then done as normal sections
%% \appendix
%% \section{}
%% \label{}

%% If you have bibdatabase file and want bibtex to generate the
%% bibitems, please use
%%
%% \bibliographystyle{elsarticle-harv} 
%% \bibliography{<your bibdatabase>}

\section{Introduction}
%\input{introduction}
\
\begin{equation}
 \frac{D \tilde{\rho}}{D \tilde{t}}
+\tilde{\rho} \tilde{\nabla}\cdot\tilde{\mathbf{u}}
 =0,
\label{eq:mass}  
\end{equation}

\begin{equation}
	\tilde{\rho} \frac{D  \tilde{Y_i}}{D \tilde{t}}
= 
	\tilde{\nabla}\cdot( \tilde{\rho} \mathcal{D}_i\tilde{\nabla} \tilde{Y_i})
	+
    s_i \omega.
\label{eq:specie}  
\end{equation}

%{\color{red}
The term $\omega$ is the reaction rate and 
$\omega_i = s_i \omega$ stands for the generation or consumption rate of species $i$.
%}

\begin{equation}
    \tilde{\rho}\frac{D  \tilde{\mathbf{u}}}{D \tilde{t}}
= - \tilde{\nabla} \tilde{p} 
  + \tilde{\nabla} \cdot \tilde{\pmb{\tau}} 
  ,
\label{eq:momentum}  
\end{equation}

Finally the energy equation is:

\begin{equation}
        \tilde{\rho}\frac{D \tilde{E} }{D \tilde{t}}
=       - 
        \tilde{\nabla} \cdot (\tilde{p} \tilde{\mathbf{u}})
        + 
        \tilde{\nabla} \cdot (\pmb{\tilde{\tau}}\cdot\tilde{\mathbf{u}}) 
        + 
        \tilde{\nabla} \cdot (\tilde{k}\tilde{\nabla} \tilde{T}) 
        + 
        \sum\limits_i \tilde{\rho}_i\tilde{U}_i\tilde{h}_i
  ,
\label{eq:E}  
\end{equation}

where 

\begin{equation}
        \sum\limits_i \tilde{\rho}_i\tilde{U}_i\tilde{h}_i
        =
        \tilde{\nabla} \cdot
        \tilde{\rho}
        \sum\limits_i 
        \mathcal{D}_{im}\tilde{\nabla}\tilde{Y}_i     
        \tilde{h}_i
\end{equation}

is the energy flux due to the diffusion of the specie i, went 
the specie i diffusing from one position to other 
carried its enthalpy, that is a form of energy. 
The chemical reaction can be occurs in the flow 
and exchange energy  due diffusion. 

A special form of energy equation can be founded if the definition of
enthalpy and the perfect gas equation are used. 


\begin{equation}
\begin{split}
        \tilde{\rho}\frac{D  \tilde{h}}{D\tilde{t}}
=
        \frac{D \tilde{p}}{D\tilde{t}}
+       
        \pmb{\tilde{\tau}}:\tilde{\nabla} \tilde{\mathbf{u}} 
+ 
        \tilde{\nabla} \cdot (\tilde{k}\tilde{\nabla} \tilde{T})
+
        \tilde{\nabla} \cdot
        \sum\limits_i 
        \tilde{\rho}\mathcal{D}_{im}\tilde{\nabla}\tilde{Y}_i     
        \tilde{h}_i.
\end{split}
\end{equation}

The absolute enthalpy is defined as:

\begin{equation}
	h=\sum_iY_ih_{i,sens}+\sum_iY_i\Delta(h_f)_i^o
\end{equation}

\begin{equation}
	\frac{\tilde{D} \tilde{h}}{\tilde{D} t}
	=
	\frac{\tilde{D}\tilde{h}_{sens}}{\tilde{D}t}
	+
	\sum_i\Delta(\tilde{h}_f)_i^o
	\frac{\tilde{D}\tilde{Y}_i}{\tilde{D}t}
\end{equation}

Comparing this equation with the total enthapy equation and 
using the equation for the conservation of the species is 
getting:

%\begin{equation}
%	\frac{\tilde{D} \tilde{h}}{\tilde{D} t}
%	=
%	\frac{\tilde{D}\tilde{h}_{sens}}{\tilde{D}t}
%	+
%	\frac{1}{\tilde{\rho}}
%	\sum_i\Delta(\tilde{h}_f)_i^o
%	\left(
%	\tilde{\nabla}\cdot( \tilde{\rho} \mathcal{D}_{im}\tilde{\nabla} \tilde{Y_i})
%	\right)
%	+
%	\frac{1}{\tilde{\rho}}
%	\sum_i\Delta(\tilde{h}_f)_i^o
%	s_i \omega.
%\end{equation}
%
%\begin{equation}
%	\tilde{\rho}
%	\frac{\tilde{D} \tilde{h}}{\tilde{D} t}
%	=
%	\tilde{\rho}
%	\frac{\tilde{D}\tilde{h}_{sens}}{\tilde{D}t}
%	+
%	\sum_i\Delta(\tilde{h}_f)_i^o
%	\left(
%	\tilde{\nabla}\cdot( \tilde{\rho} \mathcal{D}_{im}\tilde{\nabla} \tilde{Y_i})
%	\right)
%	+
%	\sum_i\Delta(\tilde{h}_f)_i^o
%	s_i \omega.
%\end{equation}
%
%\begin{equation}
%\begin{split}
%	\tilde{\rho}
%	\frac{\tilde{D}\tilde{h}_{sens}}{\tilde{D}t}
%=
%        \frac{D \tilde{p}}{D\tilde{t}}
%+       
%        \pmb{\tilde{\tau}}:\tilde{\nabla} \tilde{\mathbf{u}} 
%+ 
%        \tilde{\nabla} \cdot (\tilde{k}\tilde{\nabla} \tilde{T})
%+
%        \tilde{\nabla} \cdot
%        \sum\limits_i 
%        \tilde{\rho}\mathcal{D}_{im}\tilde{\nabla}\tilde{Y}_i     
%        \tilde{h}_i.
%	\\
%-
%	\sum_i\Delta(\tilde{h}_f)_i^o
%	\left(
%	\tilde{\nabla}\cdot( \tilde{\rho} \mathcal{D}_{im}\tilde{\nabla} \tilde{Y_i})
%	\right)
%-
%	\sum_i\Delta(\tilde{h}_f)_i^o
%	s_i \omega.
%\end{split}
%\end{equation}
%
%\begin{equation}
%\begin{split}
%	\tilde{\rho}
%	\frac{\tilde{D}\tilde{h}_{sens}}{\tilde{D}t}
%=
%        \frac{D \tilde{p}}{D\tilde{t}}
%+       
%        \pmb{\tilde{\tau}}:\tilde{\nabla} \tilde{\mathbf{u}} 
%+ 
%        \tilde{\nabla} \cdot (\tilde{k}\tilde{\nabla} \tilde{T})
%+
%        \tilde{\nabla} \cdot
%        \sum\limits_i 
%        \tilde{\rho}\mathcal{D}_{im}\tilde{\nabla}\tilde{Y}_i     
%	\tilde{h}_{sens,i}.
%+
%        \sum\limits_i 
%        \Delta(\tilde{h}_f)_i^o
%        \tilde{\nabla} \cdot
%	(
%        \tilde{\rho}\mathcal{D}_{im}\tilde{\nabla}\tilde{Y}_i     
%	)
%	\\
%-
%	\sum_i\Delta(\tilde{h}_f)_i^o
%	\left(
%	\tilde{\nabla}\cdot( \tilde{\rho} \mathcal{D}_{im}\tilde{\nabla} \tilde{Y_i})
%	\right)
%-
%	\sum_i\Delta(\tilde{h}_f)_i^o
%	s_i \omega.
%\end{split}
%\end{equation}

\begin{equation}
\begin{split}
	\tilde{\rho}
	\frac{\tilde{D}\tilde{h}_{sens}}{\tilde{D}t}
=
        \frac{D \tilde{p}}{D\tilde{t}}
+       
        \pmb{\tilde{\tau}}:\tilde{\nabla} \tilde{\mathbf{u}} 
+ 
        \tilde{\nabla} \cdot (\tilde{k}\tilde{\nabla} \tilde{T})
+
        \tilde{\nabla} \cdot
        \sum\limits_i 
        \tilde{\rho}\mathcal{D}_{im}\tilde{\nabla}\tilde{Y}_i     
	\tilde{h}_{sens,i}.
-
	\sum_i\Delta(\tilde{h}_f)_i^o
	s_i \omega.
\end{split}
\end{equation}


For the following equation will be 
considered that $h_{sens}\equiv h$


%\begin{equation}
%	h_i=\int_0^TC_{pi}dT+\Delta(h_f)_i^o
%\end{equation}


\section{Temperature form  of the Energy Equation}

%
If the gas mixture is considered as thermally perfect, 
$dh=c_{p}dT$, and assuming a mixture of perfect gases then 
%%%
%%%
\begin{equation}
\begin{split}
        \frac{d}{dt}\tilde{h}=\tilde{c}_p\frac{d}{dt}\tilde{T}
\end{split}
\end{equation}


\begin{equation}
\begin{split}
        \frac{d}{dt}\tilde{h}
        =
        \left(
                \frac{d}{dt}
		\sum\limits_i\tilde{Y}_i\tilde{h}_i
        \right)
\\
	\frac{d}{dt}\tilde{h}
        =
        \left(
                \frac{d}{dt}
		\sum\limits_i\tilde{Y}_i
		\tilde{h}_i
        \right)
\\
	\frac{d}{dt}\tilde{h}
        =
        \sum\limits_i
        \left(
                \tilde{h}_i
                \frac{d}{dt}
		\tilde{Y}_i
                +
                \tilde{Y}_i
                \frac{d}{dt}
		\tilde{h}_i
        \right)
\\
	\frac{d}{dt}\tilde{h}
        =
        \sum\limits_i
        \left(
                \tilde{h}_i
                \frac{d}{dt}
		\tilde{Y}_i
                +
                \tilde{Y}_i
                \tilde{c}_{pi}
                \frac{d}{dt}
		\tilde{T}
        \right)
\\
	\frac{\tilde{D}}{\tilde{D} t}
	\tilde{h}
        =
        \sum\limits_i
        \left(
                \tilde{h}_i
		\frac{\tilde{D}}{\tilde{D} t}
		\tilde{Y}_i
                +
                \tilde{Y}_i
                \tilde{c}_{pi}
		\frac{\tilde{D}}{\tilde{D} t}
		\tilde{T}
        \right)
\\
\text{Using the species conservation equation}
\\
	\frac{\tilde{D}}{\tilde{D} t}
	\tilde{h}
        =
        \sum\limits_i
        \left[
                \frac{\tilde{h}_i}{\tilde{\rho}}
                \tilde{\nabla}\cdot
                (
		\tilde{\rho}\mathcal{D}_{im}\tilde{\nabla}\tilde{Y}_i
		+
        	s_i \omega
		)
                +
                \tilde{Y}_i
                \tilde{c}_{pi}
		\frac{\tilde{D}}{\tilde{D} t}
		\tilde{T}
        \right]
\\
        \tilde{\rho}
	\frac{\tilde{D}}{\tilde{D} t}
	\tilde{h}
        =
        \sum\limits_i
        \left\{
                \tilde{h}_i
                [
                	\tilde{\nabla}\cdot
			   (
			\tilde{\rho}\mathcal{D}_{im}\tilde{\nabla}\tilde{Y}_i
			)
			+
        		s_i \omega
		    ]
                +
                \tilde{\rho}
                \tilde{Y}_i
                \tilde{c}_{pi}
        	\frac{D}{Dt}T
        \right\}
\end{split}
\end{equation}

\begin{equation}
\begin{split}
        \tilde{\rho}
        \frac{\tilde{D}}{\tilde{D} t}\tilde{h}
        =
        \sum\limits_i
        \left\{
                \tilde{h}_i
                [
                	\tilde{\nabla}\cdot
		     	(
			\tilde{\rho}\mathcal{D}_{im}\tilde{\nabla}\tilde{Y}_i
			)
			+
        		s_i \omega
	        	]
                +
                \tilde{\rho}
                \tilde{Y}_i
                \tilde{c}_{pi}
                \frac{\tilde{D}}{\tilde{D} t}
                \tilde{T}
        \right\}
\end{split}
\end{equation}

Remember, the energy equation for sensible enthalpy %{\color{red} (thermal part, $\int c_pdT$) } 
can be written  as:

\begin{equation}
\begin{split}
        \tilde{\rho}\frac{D  \tilde{h}}{D\tilde{t}}
=
        \frac{D \tilde{p}}{D\tilde{t}}
+       
        \pmb{\tilde{\tau}}:\tilde{\nabla} \tilde{\mathbf{u}} 
+ 
        \tilde{\nabla} \cdot (\tilde{k}\tilde{\nabla} \tilde{T})
+
        \tilde{\nabla} \cdot
        \sum\limits_i 
        \tilde{\rho}\mathcal{D}_{im}\tilde{\nabla}\tilde{Y}_i     
        \tilde{h}_i
        -
	\sum_i \Delta(\tilde{h}_f)_i^0 s_i \omega,
\end{split}
\end{equation}

\begin{equation}
       \tilde{\rho}\frac{D  \tilde{h}}{D\tilde{t}}
=
        \frac{D \tilde{p}}{D\tilde{t}}
+       
        \pmb{\tilde{\tau}}:\tilde{\nabla} \tilde{\mathbf{u}} 
+ 
        \tilde{\nabla} \cdot (\tilde{k}\tilde{\nabla} \tilde{T})
+
        \tilde{\nabla} \cdot
        \sum\limits_i 
        \tilde{\rho}\mathcal{D}_{im}\tilde{\nabla}\tilde{Y}_i     
        \tilde{h}_i
        +
        Q\omega,
\end{equation}

with

\[
   Q \equiv -\sum_i (\tilde{h}_f)_i^0  s_i 
\]

using the above equation, 

\begin{equation}
\begin{split}
        \frac{\tilde{D} \tilde{p}}{\tilde{D}\tilde{t}}
        +       
        \pmb{\tilde{\tau}}:\tilde{\nabla} \tilde{\mathbf{u}} 
        + 
        \tilde{\nabla} \cdot (\tilde{k}\tilde{\nabla} \tilde{T})
        +
        \tilde{\nabla} \cdot
        \sum\limits_i 
        \tilde{\rho}\mathcal{D}_{im}\tilde{\nabla}\tilde{Y}_i     
        \tilde{h}_i
        +
	Q\omega
        =
	\\
        \sum\limits_i
                \tilde{h}_i
                [
                	\tilde{\nabla}\cdot
			(
			\tilde{\rho}\mathcal{D}_{im}\tilde{\nabla}\tilde{Y}_i
			)
			+
        	s_i \omega
		]
        +
        \sum\limits_i
        \left(
                \tilde{\rho}
                \tilde{Y}_i
                \tilde{c}_{pi}
                \frac{\tilde{D}}{\tilde{D} t}
                \tilde{T}
        \right)
\end{split}
\end{equation}

using the conservation of species in the first term of 
the right hand side of the equation, 

\begin{equation}
\begin{split}
        \frac{\tilde{D} \tilde{p}}{\tilde{D}\tilde{t}}
        +       
        \pmb{\tilde{\tau}}:\tilde{\nabla} \tilde{\mathbf{u}} 
        + 
        \tilde{\nabla} \cdot (\tilde{k}\tilde{\nabla} \tilde{T})
        +
        \tilde{\nabla} \cdot
        \sum\limits_i 
        \tilde{\rho}\mathcal{D}_{im}\tilde{\nabla}\tilde{Y}_i     
        \tilde{h}_i
        +
	Q\omega
        =
        \tilde{\rho}
        \sum\limits_i
        \left(
                \tilde{h}_i
                \frac{D}{Dt}
		\tilde{Y}_i
	\right)
        +
        \tilde{\rho}
        \sum\limits_i
        \left(
                \tilde{Y}_i
                \tilde{c}_{pi}
                \frac{D}{Dt}
		\tilde{T}
        \right)
\end{split}
\end{equation}

defining 
\[
     \tilde{c}_p \equiv \sum\limits_i \tilde{Y}_i \tilde{c}_{pi},
\]

then

\begin{equation}
\begin{split}
        \frac{\tilde{D} \tilde{p}}{\tilde{D}\tilde{t}}
        +       
        \pmb{\tilde{\tau}}:\tilde{\nabla} \tilde{\mathbf{u}} 
        + 
        \tilde{\nabla} \cdot (\tilde{k}\tilde{\nabla} \tilde{T})
        +
        \tilde{\nabla} \cdot
        \sum\limits_i 
        \tilde{\rho}\mathcal{D}_{im}\tilde{\nabla}\tilde{Y}_i     
        \tilde{h}_i
        +
	Q\omega
        =
        \tilde{\rho}
        \sum\limits_i
        \left(
                \tilde{h}_i
                \frac{D}{Dt}
		\tilde{Y}_i
                +
		\tilde{c}_{p}
                \frac{D}{Dt}
		\tilde{T}
        \right)
\end{split}
\end{equation}


%{JHONATAN, VAMOS DEIXAR A DERIVADA TOTAL DE $Y_i$. NO FINAL TRATAMOS.
%}

Now, applying distributive propertie of the gradient to the third term  of the left hand side of the equation, 

\begin{equation}
\begin{split}
        \frac{\tilde{D} \tilde{p}}{\tilde{D}\tilde{t}}
        +       
        \pmb{\tilde{\tau}}:\tilde{\nabla} \tilde{\mathbf{u}} 
        + 
        \tilde{\nabla} \cdot (\tilde{k}\tilde{\nabla} \tilde{T})
        +
        \sum\limits_i
        \tilde{h}_i
        \tilde{\nabla}\cdot
	(
	\tilde{\rho}\mathcal{D}_{im}\tilde{\nabla}\tilde{Y}_i
	)
        +
        \sum\limits_i 
        \tilde{\rho}
        \mathcal{D}_{im}
        \tilde{\nabla}
        \tilde{Y}_i     
        \cdot
        \tilde{\nabla} 
        \tilde{h}_i
        +
	Q\omega
	\\
        =
        \sum\limits_i
        \tilde{h}_i
        [
        	\tilde{\nabla}\cdot
		(
		\tilde{\rho}\mathcal{D}_{im}\tilde{\nabla}\tilde{Y}_i
		)
		+
        	s_i \omega
	    ]
        +
        \sum\limits_i
        \left(
                \tilde{\rho}
                \tilde{Y}_i
                \tilde{c}_{pi}
                \frac{\tilde{D}}{\tilde{D} t}
                \tilde{T}
        \right)
\end{split}
\end{equation}

eliminating the equal terms, its getting  

\begin{equation}
\begin{split}
        \frac{\tilde{D} \tilde{p}}{\tilde{D}\tilde{t}}
        +       
        \pmb{\tilde{\tau}}:\tilde{\nabla} \tilde{\mathbf{u}} 
        + 
        \tilde{\nabla} \cdot (\tilde{k}\tilde{\nabla} \tilde{T})
        +
        \sum\limits_i 
        \tilde{\rho}
        \mathcal{D}_{im}
        \tilde{\nabla}
        \tilde{Y}_i     
        \cdot
        \tilde{\nabla} 
        \tilde{h}_i
	+
	Qw
        =
        \\
        \sum\limits_i
        \tilde{h}_i
        s_i  \omega
	+
        \sum\limits_i
        \left(
                \tilde{\rho}
                \tilde{Y}_i
                \tilde{c}_{pi}
                \frac{\tilde{D}}{\tilde{D} t}
                \tilde{T}
        \right)
\end{split}
\end{equation}

rearranging
\begin{equation}
\begin{split}
        \sum\limits_i
        \left(
                \tilde{\rho}
                \tilde{Y}_i
                \tilde{c}_{pi}
                \frac{\tilde{D}}{\tilde{D} t}
                \tilde{T}
        \right)
        -
        \frac{\tilde{D} \tilde{p}}{\tilde{D}\tilde{t}}
        =
        \pmb{\tilde{\tau}}:\tilde{\nabla} \tilde{\mathbf{u}} 
        + 
        \tilde{\nabla} \cdot (\tilde{k}\tilde{\nabla} \tilde{T})
        +
        \sum\limits_i 
        \tilde{\rho}
        \mathcal{D}_{im}
        \tilde{\nabla}
        \tilde{Y}_i     
        \cdot
        \tilde{\nabla} 
        \tilde{h}_i
	+
\\
	Qw
	-
        \sum\limits_i
	\tilde{h}_i
        s_i \omega	
\\
        \tilde{\rho}
        \frac{\tilde{D}}{\tilde{D} t}
        \tilde{T}
        \sum\limits_i
        \left(
                \tilde{Y}_i
                \tilde{c}_{pi}
        \right)
        -
        \frac{D \tilde{p}}{D\tilde{t}}
        =
        \pmb{\tilde{\tau}}:\tilde{\nabla} \tilde{\mathbf{u}} 
        + 
        \tilde{\nabla} \cdot (\tilde{k}\tilde{\nabla} \tilde{T})
        +
        \sum\limits_i 
        \tilde{\rho}
        \mathcal{D}_{im}
        \tilde{\nabla}
        \tilde{Y}_i     
        \cdot
        \tilde{\nabla} 
        \tilde{h}_i
	+
	Qw
	-
        \sum\limits_i
	\tilde{h}_i
        s_i \omega	
\end{split}
\end{equation}

defining 
$
{c}_{pf}
\equiv
\sum\limits_i
\left(
        \tilde{Y}_i
        \tilde{c}_{pi}
\right)
$

\begin{equation}
\begin{split}
        \tilde{\rho}
        \tilde{c}_{pf}
        \frac{\tilde{D}}{\tilde{D} t}
        \tilde{T}
        -
        \frac{\tilde{D} \tilde{p}}{\tilde{D}\tilde{t}}
        =
        \pmb{\tilde{\tau}}:\tilde{\nabla} \tilde{\mathbf{u}} 
        + 
        \tilde{\nabla} \cdot (\tilde{k}\tilde{\nabla} \tilde{T})
        +
        \sum\limits_i 
        \tilde{\rho}
        \mathcal{D}_{im}
        \tilde{\nabla}
        \tilde{Y}_i     
        \cdot
        \tilde{\nabla} 
        \tilde{h}_i
	+
	Qw
	-
        \sum\limits_i
	\tilde{h}_i
        s_i \omega	
\end{split}
\end{equation}

Using the chain rule, because the entalphy is function only of the temperature, then 

\begin{equation}
\begin{split}
        \tilde{\rho}
        \tilde{c}_{pf}
        \frac{\tilde{D}}{\tilde{D} t}
        \tilde{T}
        -
        \frac{D \tilde{p}}{D\tilde{t}}
        =
        \pmb{\tilde{\tau}}:\tilde{\nabla} \tilde{\mathbf{u}} 
        + 
        \tilde{\nabla} \cdot (\tilde{k}\tilde{\nabla} \tilde{T})
        +
        \sum\limits_i 
        \tilde{\rho}
        \mathcal{D}_{im}
        \tilde{\nabla}
        \tilde{Y}_i     
        \cdot
        \left(
        \frac{d \tilde{h}_i}{ d T}
        \tilde{\nabla}T 
        \right)
	+
	Qw
	-
        \sum\limits_i
	\tilde{h}_i
        s_i \omega	
\\
        \tilde{\rho}
        \tilde{c}_{pf}
        \frac{\tilde{D}}{\tilde{D} t}
        \tilde{T}
        -
        \frac{D \tilde{p}}{D\tilde{t}}
        =
        \pmb{\tilde{\tau}}:\tilde{\nabla} \tilde{\mathbf{u}} 
        + 
        \tilde{\nabla} \cdot (\tilde{k}\tilde{\nabla} \tilde{T})
        +
        \sum\limits_i 
        \tilde{\rho}
        \mathcal{D}_{im}
        \tilde{\nabla}
        \tilde{Y}_i     
        \cdot
        \left(
        c_{pi}
        \tilde{\nabla}T 
        \right)
	+
	Qw
	-
        \sum\limits_i
	\tilde{h}_i
        s_i \omega	
%      
%	- \sum\limits_i  \tilde{h}_i  s_i	  \omega
\end{split}
\end{equation}

{\color{green}
Não entendi isto, e a mesma coisa que eu faço embaixo 
}
%
%{\color{red}
%\[
%         \tilde{\rho} \tilde{c}_{p}
%        \frac{D}{Dt}
%		\tilde{T}
%		-
%		 \frac{\tilde{D} \tilde{p}}{\tilde{D}\tilde{t}}
%		=
%		\tilde{\nabla} \cdot (\tilde{k}\tilde{\nabla} \tilde{T})
%                +
%        {\color{red} Q\omega}
%        +
%\]
%\[
%        \left[
%        \pmb{\tilde{\tau}}:\tilde{\nabla} \tilde{\mathbf{u}} 
%        -
%         \tilde{\rho}
%        \sum\limits_i
%        \left(
%                \tilde{h}_i
%                \frac{D}{Dt}
%		\tilde{Y}_i
%		\right)
%        +
%        \tilde{\nabla} \cdot
%        \sum\limits_i 
%        \tilde{\rho}\mathcal{D}_{im}\tilde{\nabla}\tilde{Y}_i     
%        \tilde{h}_i
%        \right]
%\]
%
%\[
%         \tilde{\rho} \tilde{c}_{p}
%         \frac{\tilde{D} } {\tilde{D}\tilde{t}}
%		(\frac{ \tilde{p} }{ \tilde{R} \tilde{\rho} } )
%		-
%		 \frac{\tilde{D} \tilde{p}}{\tilde{D}\tilde{t}}
%		=
%		\tilde{\nabla} \cdot 
%		\left[\tilde{k}\tilde{\nabla} (\frac{ \tilde{p} }{ \tilde{R} \tilde{\rho} } )\right]
%                +
%        {\color{red} Q\omega}
%        +
%\]
%\[
%         \tilde{\rho} \frac{ \tilde{c}_{p} }{ \tilde{R} }
%         \left(
%         \frac{ 1 }{  \tilde{\rho} }
%         \frac{\tilde{D}  \tilde{p}} {\tilde{D}\tilde{t}}
%		-
%		(\frac{ \tilde{p} }{  \tilde{\rho}^2 } )
%		 \frac{\tilde{D}\tilde{\rho} } {\tilde{D}\tilde{t}}
%		\right)
%		-
%		 \frac{\tilde{D} \tilde{p}}{\tilde{D}\tilde{t}}
%		=
%		\tilde{\nabla} \cdot
%		\left[ \tilde{\rho} 
%		       \frac{\tilde{k}}{\tilde{\rho} \tilde{c}_{p} }
%		       \frac{\tilde{c}_{p}}{\tilde{R}}
%		\tilde{\nabla} (\frac{ \tilde{p} }{  \tilde{\rho} } )\right]
%                +
%        {\color{red} Q\omega}
%        +
%\]
%\[
%          \frac{ \tilde{c}_{p} }{ \tilde{R} }
%         \left(
%         \frac{\tilde{D}  \tilde{p}} {\tilde{D}\tilde{t}}
%		-
%		(\frac{ \tilde{p} }{  \tilde{\rho} } )
%		 \frac{\tilde{D}\tilde{\rho} } {\tilde{D}\tilde{t}}
%		\right)
%		-
%		 \frac{\tilde{D} \tilde{p}}{\tilde{D}\tilde{t}}
%		=
%		\frac{\tilde{c}_{p}}{\tilde{R}}
%		\tilde{\nabla} \cdot 
%			\left[ \tilde{\rho} \alpha
%		\tilde{\nabla} (\frac{ \tilde{p} }{  \tilde{\rho} } )\right]
%                +
%        {\color{red} Q\omega}
%        +
%\]
%\[
%        \left[
%        \pmb{\tilde{\tau}}:\tilde{\nabla} \tilde{\mathbf{u}} 
%        -
%        \sum\limits_i
%        \left(
%                 \tilde{\rho}\tilde{h}_i
%                \frac{D}{Dt}
%		\tilde{Y}_i
%		\right)
%        +
%        \tilde{\nabla} \cdot
%        \sum\limits_i 
%        \tilde{\rho}\mathcal{D}_{im}\tilde{\nabla}\tilde{Y}_i     
%        \tilde{h}_i
%        \right]
%\]
%
%\[
%          \left(
%          \frac{ \tilde{c}_{p} }{ \tilde{R} }
%          -
%          1
%          \right)
%         \frac{\tilde{D}  \tilde{p}} {\tilde{D}\tilde{t} }
%		=
%		\frac{\tilde{c}_{p}}{\tilde{R}}
%		\tilde{\nabla} \cdot 
%			\left[ \tilde{\rho} \alpha
%		\tilde{\nabla} (\frac{ \tilde{p} }{  \tilde{\rho} } )\right]
%                +
%        {\color{red} Q\omega}
%        +
%\]
%\[
%        \left[
%        \frac{ \tilde{c}_{p} }{ \tilde{R} }
%        (\frac{ \tilde{p} }{  \tilde{\rho} } )
%		 \frac{\tilde{D}\tilde{\rho} } {\tilde{D}\tilde{t}}
%		 +
%        \pmb{\tilde{\tau}}:\tilde{\nabla} \tilde{\mathbf{u}} 
%        -
%        \sum\limits_i
%        \left(
%                 \tilde{\rho}\tilde{h}_i
%                \frac{D}{Dt}
%		\tilde{Y}_i
%		\right)
%        +
%        \tilde{\nabla} \cdot
%        \sum\limits_i 
%        \tilde{\rho}\mathcal{D}_{im}\tilde{\nabla}\tilde{Y}_i     
%        \tilde{h}_i
%        \right]
%\]
%
%
%\[
%          \left(
%          \frac{ \tilde{c}_{p} }{ \tilde{R} }
%          -
%          1
%          \right)
%         \frac{\tilde{D}  \tilde{p}} {\tilde{D}\tilde{t} }
%		=
%          \left(
%          \frac{ \tilde{c}_{p} }{ \tilde{R} }
%          -
%          1
%          \right)
%		\tilde{\nabla} \cdot 
%			\left[ \tilde{\rho} \alpha
%		\tilde{\nabla} (\frac{ \tilde{p} }{  \tilde{\rho} } )\right]
%                +
%        {\color{red} Q\omega}
%        +
%\]
%\[
%        \left[
%        \tilde{\nabla} \cdot 
%		\left( 
%		\tilde{\rho} \alpha
%		\tilde{\nabla} (\frac{ \tilde{p} }{  \tilde{\rho} } )
%		\right)
%        +
%        \frac{ \tilde{c}_{p} }{ \tilde{R} }
%        (\frac{ \tilde{p} }{  \tilde{\rho} } )
%		 \frac{\tilde{D}\tilde{\rho} } {\tilde{D}\tilde{t}}
%		 +
%        \pmb{\tilde{\tau}}:\tilde{\nabla} \tilde{\mathbf{u}} 
%        -
%        \sum\limits_i
%        \left(
%                 \tilde{\rho}\tilde{h}_i
%                \frac{D}{Dt}
%		\tilde{Y}_i
%		\right)
%        +
%        \tilde{\nabla} \cdot
%        \sum\limits_i 
%        \tilde{\rho}\mathcal{D}_{im}\tilde{\nabla}\tilde{Y}_i     
%        \tilde{h}_i
%        \right]
%\]
%
%\[
%          \frac{ 1 }{ \gamma - 1}
%         \frac{\tilde{D}  \tilde{p}} {\tilde{D}\tilde{t} }
%		=
%        \frac{ 1 }{ \gamma -1 }
%		\tilde{\nabla} \cdot 
%			\left[ \tilde{\rho} \alpha
%		\tilde{\nabla} (\frac{ \tilde{p} }{  \tilde{\rho} } )\right]
%                +
%        {\color{red} Q\omega}
%        +
%\]
%\[
%        \left[
%        \tilde{\nabla} \cdot 
%		\left( 
%		\tilde{\rho} \alpha
%		\tilde{\nabla} (\frac{ \tilde{p} }{  \tilde{\rho} } )
%		\right)
%        +
%        \frac{ \gamma }{ \gamma - 1 }
%        (\frac{ \tilde{p} }{  \tilde{\rho} } )
%		 \frac{\tilde{D}\tilde{\rho} } {\tilde{D}\tilde{t}}
%		 +
%        \pmb{\tilde{\tau}}:\tilde{\nabla} \tilde{\mathbf{u}} 
%        -
%        \sum\limits_i
%        \left(
%                 \tilde{\rho}\tilde{h}_i
%                \frac{D}{Dt}
%		\tilde{Y}_i
%		\right)
%        +
%        \tilde{\nabla} \cdot
%        \sum\limits_i 
%        \tilde{\rho}\mathcal{D}_{im}\tilde{\nabla}\tilde{Y}_i     
%        \tilde{h}_i
%        \right]
%\]
%
%\newpage
%
%\[
%         \frac{\tilde{D}  \tilde{p}} {\tilde{D}\tilde{t} }
%         =
%         \tilde{\rho}  
%         \frac{\tilde{D} } {\tilde{D}\tilde{t} }
%         \frac{ \tilde{p}}{ \tilde{\rho}}
%         +
%         \frac{ \tilde{p}}{ \tilde{\rho}}
%         \frac{\tilde{D} \tilde{\rho}} {\tilde{D}\tilde{t} }
%        = 
%		\tilde{\nabla} \cdot 
%			\left[ \tilde{\rho} \alpha
%		\tilde{\nabla} (\frac{ \tilde{p} }{  \tilde{\rho} } )\right]
%                +
%        (\gamma - 1) Q \omega
%        +
%\]
%\[
%        (\gamma - 1)
%        \left[
%        \tilde{\nabla} \cdot 
%		\left( 
%		\tilde{\rho} \alpha
%		\tilde{\nabla} (\frac{ \tilde{p} }{  \tilde{\rho} } )
%		\right)
%        +
%        \frac{ \gamma }{ \gamma - 1 }
%        (\frac{ \tilde{p} }{  \tilde{\rho} } )
%		 \frac{\tilde{D}\tilde{\rho} } {\tilde{D}\tilde{t}}
%		 +
%        \pmb{\tilde{\tau}}:\tilde{\nabla} \tilde{\mathbf{u}} 
%        -
%        \sum\limits_i
%        \left(
%                 \tilde{\rho}\tilde{h}_i
%                \frac{D}{Dt}
%		\tilde{Y}_i
%		\right)
%        +
%        \tilde{\nabla} \cdot
%        \sum\limits_i 
%        \tilde{\rho}\mathcal{D}_{im}\tilde{\nabla}\tilde{Y}_i     
%        \tilde{h}_i
%        \right]
%\]
%
%\[
%         \tilde{\rho}  
%         \frac{\tilde{D} } {\tilde{D}\tilde{t} }
%         \frac{ \tilde{p}}{ \tilde{\rho}}
%        = 
%		\tilde{\nabla} \cdot 
%			\left[ \tilde{\rho} \alpha
%		\tilde{\nabla} (\frac{ \tilde{p} }{  \tilde{\rho} } )\right]
%                +
%        (\gamma - 1) Q \omega
%        +
%\]

%\[
%        (\gamma - 1)
%        \left[
%        \tilde{\nabla} \cdot 
%		\left( 
%		\tilde{\rho} \alpha
%		\tilde{\nabla} (\frac{ \tilde{p} }{  \tilde{\rho} } )
%		\right)
%        +
%        (\frac{ \tilde{p} }{  \tilde{\rho} } )
%		 \frac{\tilde{D}\tilde{\rho} } {\tilde{D}\tilde{t}}
%		 +
%        \pmb{\tilde{\tau}}:\tilde{\nabla} \tilde{\mathbf{u}} 
%        -
%        \sum\limits_i
%        \left(
%                 \tilde{\rho}\tilde{h}_i
%                \frac{D}{Dt}
%		\tilde{Y}_i
%		\right)
%        +
%        \tilde{\nabla} \cdot
%        \sum\limits_i 
%        \tilde{\rho}\mathcal{D}_{im}\tilde{\nabla}\tilde{Y}_i     
%        \tilde{h}_i
%        \right]
%\]
%Since
%\[   \frac{1} {  \tilde{\rho} } 
%		 \frac{\tilde{D}\tilde{\rho} } {\tilde{D}\tilde{t}}
%	 =
%	 - \tilde{\nabla} \cdot \tilde{\mathbf{u}}
%\]
%

\begin{equation}
         \tilde{\rho}  
         \frac{\tilde{D} } {\tilde{D}\tilde{t} }
         (\frac{ \tilde{p}}{ \tilde{\rho}})
        = 
		\tilde{\nabla} \cdot 
			\left[ \tilde{\rho} \alpha
		\tilde{\nabla} (\frac{ \tilde{p} }{  \tilde{\rho} } )\right]
                +
        (\gamma - 1) Q \omega
        +
        F
\label{eqp}
\end{equation}

\begin{equation}
    F \equiv 
            (\gamma - 1)
        \left[
        \tilde{\nabla} \cdot 
		\left( 
		\tilde{\rho} \alpha
		\tilde{\nabla} (\frac{ \tilde{p} }{  \tilde{\rho} } )
		\right)
        -
        \tilde{p}  \tilde{\nabla} \cdot \tilde{\mathbf{u}}
		 +
        \pmb{\tilde{\tau}}:\tilde{\nabla} \tilde{\mathbf{u}} 
        -
        \sum\limits_i
        \left(
                 \tilde{\rho}\tilde{h}_i
                \frac{D}{Dt}
		\tilde{Y}_i
		\right)
        +
        \tilde{\nabla} \cdot
        \sum\limits_i 
        \tilde{\rho}\mathcal{D}_{im}\tilde{\nabla}\tilde{Y}_i     
        \tilde{h}_i
        \right]
\label{eqF}
\end{equation}
%}

Enthalpy of a chemically reacting mixture  
is define as $h=\sum_iY_ih_i$, and 
the specific heat  at constant pressure is 
$C_p=\left(\partial{h}/\partial{T}\right)_p$,
then:

\begin{equation}
\begin{split}
C_p
=
\left(\frac{\partial{h}}{\partial{T}}\right)_p
=
\left(\frac{\partial}{\partial{T}}\sum_i Y_ih_i\right)_p
=
\left(\sum_i Y_i\frac{\partial{h_i}}{\partial{T}}\right)_p
+
\left(\sum_i h_i\frac{\partial{Y_i}}{\partial{T}}\right)_p
\end{split}
\end{equation}

Defining: 

\begin{equation}
C_{pi}=\frac{\partial{h_i}}{\partial{T}}
\end{equation}

as the specific heat of the specie $i$.

\begin{equation}
\begin{split}
C_p
=
\left(\sum_i Y_iC_{pi}\right)_p
+
\left(\sum_i h_i\frac{\partial{Y_i}}{\partial{T}}\right)_p
\end{split}
\end{equation}

\begin{equation}
C_{pf}=\sum_iY_iC_{pi}
\end{equation}

\begin{equation}
\begin{split}
C_p
=
\underbrace{
C_{pf}
	    }_{\text{frozem}C_p}
+
\underbrace{
		\left(\sum_i h_i\frac{\partial{Y_i}}{\partial{T}}\right)_p
}_{\text{chemical reacting} C_p}
\end{split}
\end{equation}


\section{Pressure form of the energy equation}


Using the ideal gas equation:

\begin{equation}
        \tilde{T}=\frac{\tilde{p}}{\tilde{\rho}\tilde{R}}
\end{equation}

and the definition of $R$ for a chemical reacting flow 

\begin{equation}
	\tilde{R}=\sum_iY_i\tilde{R}_i
\end{equation}

\begin{equation}
\begin{split}
        \tilde{\rho}
        \tilde{c}_{pf}
        \frac{\tilde{D}}{\tilde{D} t}
        \left(
                \frac{\tilde{p}}{\tilde{\rho}\tilde{R}}
        \right)
        -
        \frac{D \tilde{p}}{D\tilde{t}}
        =
        \pmb{\tilde{\tau}}:\tilde{\nabla} \tilde{\mathbf{u}} 
        + 
        \tilde{\nabla} \cdot (\tilde{k}\tilde{\nabla} \tilde{T})
        +
        \sum\limits_i 
        \tilde{\rho}
        \mathcal{D}_{im}
        \tilde{\nabla}
        \tilde{Y}_i     
        \cdot
        \left(
                c_{pi}
                \tilde{\nabla}
                \left(
                	\frac{\tilde{p}}{\tilde{\rho}\tilde{R}}
        	\right)
        \right)
	+
	Qw
	-
        \sum\limits_i
	\tilde{h}_i
        s_i \omega	
\end{split}
\end{equation}

\begin{equation}
\begin{split}
        \tilde{\rho}
        \tilde{c}_{pf}
        \tilde{p} 
        \frac{\tilde{D}}{\tilde{D} t}
        \left(
                \frac{1}{\tilde{\rho}\tilde{R}}
        \right)
	+
        \frac{1}{\tilde{R}}
        \tilde{c}_{pf}
        \frac{\tilde{D}}{\tilde{D} t}
        \left(
                \tilde{p}
        \right)
        -
        \frac{D \tilde{p}}{D\tilde{t}}
        =
\\
        \pmb{\tilde{\tau}}:\tilde{\nabla} \tilde{\mathbf{u}} 
        + 
        \tilde{\nabla} \cdot (\tilde{k}\tilde{\nabla} \tilde{T})
        +
        \sum\limits_i 
        \tilde{\rho}
        \mathcal{D}_{im}
        \tilde{\nabla}
        \tilde{Y}_i     
        \cdot
        \left(
                c_{pi}
                \tilde{\nabla}
                \left(
                	\frac{\tilde{p}}{\tilde{\rho}\tilde{R}}
                \right)
        \right)
	+
	Qw
	-
        \sum\limits_i
	\tilde{h}_i
        s_i \omega	
\\
\\
        \tilde{\rho}
        \tilde{c}_{pf}
        \tilde{p}
        \frac{\tilde{D}}{\tilde{D} t}
        \left(
                \frac{1}{\tilde{\rho}\tilde{R}}
        \right)
	+
        \left(
        	\frac{\tilde{c}_{pf}}{\tilde{R}}
		-1
        \right)
        \frac{D \tilde{p}}{D\tilde{t}}
        =
	\\
        \pmb{\tilde{\tau}}:\tilde{\nabla} \tilde{\mathbf{u}} 
        + 
        \tilde{\nabla} \cdot (\tilde{k}\tilde{\nabla} \tilde{T})
        +
        \sum\limits_i 
        \tilde{\rho}
        \mathcal{D}_{im}
        \tilde{\nabla}
        \tilde{Y}_i     
        \cdot
        \left(
                c_{pi}
                \tilde{\nabla}
                \left(
                	\frac{\tilde{p}}{\tilde{\rho}\tilde{R}}
                \right)
        \right)
	+
	Qw
	-
        \sum\limits_i
	\tilde{h}_i
        s_i \omega	
\end{split}
\end{equation}

using the definition of $\gamma=\dfrac{C_p}{C_v}$ is 
posible to write the  following terms as:

\begin{equation}
\begin{split}
	\frac{\tilde{c}_{pf}}{\tilde{R}}=\frac{\tilde{\gamma}}{\tilde{\gamma-1}}
	\\
        \left(
        	\frac{\tilde{c}_{pf}}{\tilde{R}}
		-1
        \right)
	=
	%R=cp-cv
	%R/cp=1-1/gamma
	%R/cp=1-1/gamma
	%R/cp=(gamma-1)/gamma
	%Cp/R=gamma/(gamma-1)-1
	%Cp/R=(gamma-(gamma-1))/(gamma-1)
	%Cp/R=(1)/(gamma-1)
        \left(
		\frac{1}{\tilde{\gamma}-1}
        \right),
	\label{cp}
\end{split}
\end{equation}

using its in the energy equation:  

\begin{equation}
\begin{split}
        \frac{D \tilde{p}}{D\tilde{t}}
	+
        \tilde{\rho}
	(\tilde{\gamma}-1)
        \tilde{c}_{pf}
        \tilde{p} 
        \frac{\tilde{D}}{\tilde{D} t}
        \left(
                \frac{1}{\tilde{\rho}\tilde{R}}
        \right)
        =
	(\tilde{\gamma}-1)
        \pmb{\tilde{\tau}}:\tilde{\nabla} \tilde{\mathbf{u}} 
	+
	(\tilde{\gamma}-1)
        \tilde{\nabla} \cdot (\tilde{k}\tilde{\nabla} \tilde{T})
        +
	\\
	(\tilde{\gamma}-1)
        \sum\limits_i 
        \tilde{\rho}
        \mathcal{D}_{im}
        \tilde{\nabla}
        \tilde{Y}_i     
        \cdot
        \left(
                c_{pi}
                \tilde{\nabla}
                \left(
                	\frac{\tilde{p}}{\tilde{\rho}\tilde{R}}
                \right)
        \right)
	+
	(\tilde{\gamma}-1)
	Qw
	-
	(\tilde{\gamma}-1)
        \sum\limits_i
	\tilde{h}_i
        s_i \omega	
\end{split}
\end{equation}

\begin{equation}
\begin{split}
        \frac{D \tilde{p}}{D\tilde{t}}
        %\tilde{\rho}
	%(\tilde{\gamma}-1)
        %\tilde{c}_{pf}
        %\frac{\tilde{D}}{\tilde{D} t}
	%(\tilde{\rho}\tilde{R})^{-1}
	-
	(\tilde{\gamma}-1)
        \tilde{c}_{pf}
	\frac{ \tilde{p} }{\tilde{\rho}\tilde{R}^2}
        \frac{\tilde{D}}{\tilde{D} t}
	(
	\tilde{\rho}\tilde{R}
	)
        =
	(\tilde{\gamma}-1)
        \pmb{\tilde{\tau}}:\tilde{\nabla} \tilde{\mathbf{u}} 
        + 
	(\tilde{\gamma}-1)
        \tilde{\nabla} \cdot (\tilde{k}\tilde{\nabla} \tilde{T})
        +
	\\
	(\tilde{\gamma}-1)
        \sum\limits_i 
        \tilde{\rho}
        \mathcal{D}_{im}
        \tilde{\nabla}
        \tilde{Y}_i     
        \cdot
        \left(
                c_{pi}
                \tilde{\nabla}
                \left(
                	\frac{\tilde{p}}{\tilde{\rho}\tilde{R}}
                \right)
        \right)
	+
	(\tilde{\gamma}-1)
	Qw
	-
	(\tilde{\gamma}-1)
        \sum\limits_i
	\tilde{h}_i
        s_i \omega	
\end{split}
\end{equation}
opening the derivative of the second term of rhs, then 


\begin{equation}
\begin{split}
        \frac{D \tilde{p}}{D\tilde{t}}
	-
	(\tilde{\gamma}-1)
        \tilde{c}_{pf}
	\frac{ \tilde{p} }{\tilde{\rho}\tilde{R}}
        \frac{\tilde{D}}{\tilde{D} t}
	(
	\tilde{\rho}
	)
	-
	(\tilde{\gamma}-1)
        \tilde{c}_{pf}
	\frac{  \tilde{p} } {\tilde{R}^2}
        \frac{\tilde{D}}{\tilde{D} t}
	\tilde{R}
        =
	\\
	(\tilde{\gamma}-1)
        \pmb{\tilde{\tau}}:\tilde{\nabla} \tilde{\mathbf{u}} 
        + 
	(\tilde{\gamma}-1)
        \tilde{\nabla} \cdot (\tilde{k}\tilde{\nabla} \tilde{T})
        +
	\\
	(\tilde{\gamma}-1)
        \sum\limits_i 
        \tilde{\rho}
        \mathcal{D}_{im}
        \tilde{\nabla}
        \tilde{Y}_i     
        \cdot
        \left(
                c_{pi}
                \tilde{\nabla}
                \left(
                	\frac{\tilde{p}}{\tilde{\rho}\tilde{R}}
                \right)
        \right)
	+
	(\tilde{\gamma}-1)
	Qw
	-
	(\tilde{\gamma}-1)
        \sum\limits_i
	\tilde{h}_i
        s_i \omega	
\end{split}
\end{equation}


Considering that $\tilde{R}$ is constant to simpliefied the problem,

\begin{equation}
\begin{split}
        \frac{D \tilde{p}}{D\tilde{t}}
	-
	(\tilde{\gamma}-1)
	\frac{\tilde{c}_{pf} \tilde{p}   }{\tilde{\rho}\tilde{R}}
        \frac{\tilde{D}}{\tilde{D} t}
	(
	\tilde{\rho}
	)
        =
	(\tilde{\gamma}-1)
        \pmb{\tilde{\tau}}:\tilde{\nabla} \tilde{\mathbf{u}} 
        + 
	(\tilde{\gamma}-1)
        \tilde{\nabla} \cdot (\tilde{k}\tilde{\nabla} \tilde{T})
        +
	\\
	(\tilde{\gamma}-1)
        \sum\limits_i 
        \tilde{\rho}
        \mathcal{D}_{im}
        \tilde{\nabla}
        \tilde{Y}_i     
        \cdot
        \left(
                c_{pi}
                \tilde{\nabla}
                \left(
                	\frac{\tilde{p}}{\tilde{\rho}\tilde{R}}
                \right)
        \right)
	+
	(\tilde{\gamma}-1)
	Qw
	-
	(\tilde{\gamma}-1)
        \sum\limits_i
	\tilde{h}_i
        s_i \omega	
\end{split}
\end{equation}

Using the conservation mass equation 
to eliminate $\dfrac{\tilde{D}\tilde{\rho}}{\tilde{D}t}$

\begin{equation}
\begin{split}
        \frac{D \tilde{p}}{D\tilde{t}}
	-
	(\tilde{\gamma}-1)
	\frac{\tilde{c}_{pf} \tilde{p} }  {\tilde{R}}
	\tilde{\nabla}\cdot{\mathbf{u}}
        =
	(\tilde{\gamma}-1)
        \pmb{\tilde{\tau}}:\tilde{\nabla} \tilde{\mathbf{u}} 
        + 
	(\tilde{\gamma}-1)
        \tilde{\nabla} \cdot (\tilde{k}\tilde{\nabla} \tilde{T})
        +
	\\
	(\tilde{\gamma}-1)
        \sum\limits_i 
        \tilde{\rho}
        \mathcal{D}_{im}
        \tilde{\nabla}
        \tilde{Y}_i     
        \cdot
        \left(
                c_{pi}
                \tilde{\nabla}
                \left(
                	\frac{\tilde{p}}{\tilde{\rho}\tilde{R}}
                \right)
        \right)
	+
	(\tilde{\gamma}-1)
	Qw
	-
	(\tilde{\gamma}-1)
        \sum\limits_i
	\tilde{h}_i
        s_i \omega	
\end{split}
\end{equation}

and the relation of $C_p$ \ref{cp} 

\begin{equation}
\begin{split}
        \frac{D \tilde{p}}{D\tilde{t}}
	-
	\tilde{\gamma}
	\tilde{p} 
	\tilde{\nabla}\cdot{\mathbf{u}}
        =
	(\tilde{\gamma}-1)
        \pmb{\tilde{\tau}}:\tilde{\nabla} \tilde{\mathbf{u}} 
        + 
	(\tilde{\gamma}-1)
        \tilde{\nabla} \cdot (\tilde{k}\tilde{\nabla} \tilde{T})
        +
	\\
	(\tilde{\gamma}-1)
        \sum\limits_i 
        \tilde{\rho}
        \mathcal{D}_{im}
        \tilde{\nabla}
        \tilde{Y}_i     
        \cdot
        \left(
                c_{pi}
                \tilde{\nabla}
                \left(
                	\frac{\tilde{p}}{\tilde{\rho}\tilde{R}}
                \right)
        \right)
	+
	(\tilde{\gamma}-1)
	Qw
	-
	(\tilde{\gamma}-1)
        \sum\limits_i
	\tilde{h}_i
        s_i \omega	
\end{split}
\end{equation}


\section{Zel'dovich formulation}

Zel'dovich formulation to difussive  flame, $Da\to \infty$.

Multiply the species equation of the fuel by the $s_O$ 

\begin{equation}
	s_O
	\tilde{\rho} \frac{D  \tilde{Y}_F }{D \tilde{t}}
= 
	s_O
	\tilde{\nabla}\cdot( \tilde{\rho} \mathcal{D}_f\tilde{\nabla} \tilde{Y}_F)
	+
	s_O
        s_F w,
\label{eqYF}
\end{equation}


and the oxidant species equation by $s_F$ 
\begin{equation}
	s_F
	\tilde{\rho} \frac{D  \tilde{Y}_O }{D \tilde{t}}
= 
	s_F
	\tilde{\nabla}\cdot( \tilde{\rho} \mathcal{D}_o\tilde{\nabla} 
	\tilde{Y}_O
	)
	+
	s_F
    	s_O 
	w,
\label{eqYO}
\end{equation}


Subtracting  the above equations,

\begin{equation}
	\tilde{\rho} \frac{D  }{D \tilde{t}} 
	(s_O \tilde{Y}_F - s_f \tilde{Y}_O + 1) 
= 
	\tilde{\nabla}
	\cdot
	\left[
		\tilde{\rho} 
		\left( 
			\mathcal{D}_f\tilde{\nabla} s_o \tilde{Y}_F)
			-
		    \mathcal{D}_o\tilde{\nabla} {s_f \tilde{Y}_O}
		\right) 
	\right] 
\end{equation}

Assuming that $\mathcal{D}_f=\mathcal{D}_f=\mathcal{D}$ and 
defining the mixture fraction function $Z$ as 

\begin{equation}
Z 
\equiv 
s_O\hat{Y}_F 
- 
s_F \hat{Y}_O 
+ 1 
= 
S\hat{Y}_F 
- 
\hat{Y}_O 
+ 
1,
\end{equation}

since $s_F=1$ and $s_O= S \equiv s Y_{F\infty}/Y_{O-\infty}$.
%

\begin{equation}
	\tilde{\rho} \frac{D Z }{D \tilde{t}}
= 
	\tilde{\nabla}
	\cdot
	\left(
		\tilde{\rho} \mathcal{D} \tilde{\nabla} Z 
	\right)  
\end{equation}

Multiplied the pressure energy equation by 
$(s_F+s_O)/[\tilde{Q}(\tilde{\gamma}-1]$

\begin{equation}
\begin{split}
	\frac{(s_F+s_O)}{[\tilde{Q}(\tilde{\gamma}-1)]}
	\left(
        	\frac{D \tilde{p}}{D\tilde{t}}
		-
		\tilde{\gamma}
		\tilde{p} 
		\tilde{\nabla}\cdot{\mathbf{u}}
	\right)
        =
	\frac{(s_F+s_O)}{\tilde{Q}}
        \pmb{\tilde{\tau}}:\tilde{\nabla} \tilde{\mathbf{u}} 
        + 
	\frac{(s_F+s_O)}{\tilde{Q}}
        \tilde{\nabla} \cdot (\tilde{k}\tilde{\nabla} \tilde{T})
        +
	\\
	\frac{(s_F+s_O)}{\tilde{Q}}
        \sum\limits_i 
        \tilde{\rho}
        \mathcal{D}_{im}
        \tilde{\nabla}
        \tilde{Y}_i     
        \cdot
        \left(
                c_{pi}
                \tilde{\nabla}
                \left(
                	\frac{\tilde{p}}{\tilde{\rho}\tilde{R}}
                \right)
        \right)
	+
	(s_F+s_O)w
	-
	\frac{(s_F+s_O)}{\tilde{Q}}
        \sum\limits_i
	\tilde{h}_i
        s_i \omega	
\end{split}
\end{equation}

now adding the species consevation equation and subtracting it for the above equation to 
eliminate the chemical reaction term is obtained  

\begin{equation}
\begin{split}
	\frac{(s_F+s_O)}{[\tilde{Q}(\tilde{\gamma}-1)]}
	\left(
        	\frac{D \tilde{p}}{D\tilde{t}}
		-
		\tilde{\gamma}
		\tilde{p} 
		\tilde{\nabla}\cdot{\mathbf{u}}
	\right)
	+
	\tilde{\rho} \frac{D  \tilde{Y}_F }{D \tilde{t}}
	+
	\tilde{\rho} \frac{D  \tilde{Y}_O }{D \tilde{t}}
        =
	\\
	\frac{(s_F+s_O)}{\tilde{Q}}
        \pmb{\tilde{\tau}}:\tilde{\nabla} \tilde{\mathbf{u}} 
        - 
	\frac{(s_F+s_O)}{\tilde{Q}}
        \tilde{\nabla} \cdot (\tilde{k}\tilde{\nabla} \tilde{T})
	\\
	\frac{(s_F+s_O)}{\tilde{Q}}
        \sum\limits_i 
        \tilde{\rho}
        \mathcal{D}_{im}
        \tilde{\nabla}
        \tilde{Y}_i     
        \cdot
        \left(
                c_{pi}
                \tilde{\nabla}
                \left(
                	\frac{\tilde{p}}{\tilde{\rho}\tilde{R}}
                \right)
        \right)
	-
	\frac{(s_F+s_O)}{\tilde{Q}}
        \sum\limits_i
	\tilde{h}_i
        s_i \omega	
	-
	\tilde{\nabla}\cdot( \tilde{\rho} \mathcal{D}_f\tilde{\nabla} \tilde{Y}_F)
	-
	\tilde{\nabla}\cdot( \tilde{\rho} \mathcal{D}_f\tilde{\nabla} \tilde{Y}_O)
\end{split}
\end{equation}

\begin{equation}
\begin{split}
	\frac{(s_F+s_O)}{[\tilde{Q}(\tilde{\gamma}-1)]}
	\left(
        	\frac{D \tilde{p}}{D\tilde{t}}
		\textcolor{red}{+}
		\tilde{\gamma}
		\tilde{p} 
		\tilde{\nabla}\cdot{\mathbf{u}}
	\right)
	+
	\tilde{\rho} \frac{D  }{D \tilde{t}}(\tilde{Y}_F+\tilde{Y}_O)
        =
	\\
	\frac{(s_F+s_O)}{\tilde{Q}}
        \pmb{\tilde{\tau}}:\tilde{\nabla} \tilde{\mathbf{u}} 
        - 
	\frac{(s_F+s_O)}{\tilde{Q}}
        \tilde{\nabla} \cdot (\tilde{k}\tilde{\nabla} \tilde{T})
	\\
	\frac{(s_F+s_O)}{\tilde{Q}}
        \sum\limits_i 
        \tilde{\rho}
        \mathcal{D}_{im}
        \tilde{\nabla}
        \tilde{Y}_i     
        \cdot
        \left(
                c_{pi}
                \tilde{\nabla}
                \left(
                	\frac{\tilde{p}}{\tilde{\rho}\tilde{R}}
                \right)
        \right)
	-
	\frac{(s_F+s_O)}{\tilde{Q}}
        \sum\limits_i
	\tilde{h}_i
        s_i \omega	
	-
	\tilde{\nabla}\cdot( \tilde{\rho} \mathcal{D}_f\tilde{\nabla} (\tilde{Y}_F+\tilde{Y}_O))
\end{split}
\end{equation}

using the mass conservation equation as:

\begin{equation}
\begin{split}
	\frac{(s_F+s_O)}{[\tilde{Q}(\tilde{\gamma}-1)]}
	\left(
        	\frac{D \tilde{p}}{D\tilde{t}}
		+
		\tilde{\gamma}
		\tilde{p} 
		\tilde{\nabla}\cdot{\mathbf{u}}
	\right)
	+
	\tilde{\rho} \frac{D  }{D \tilde{t}}(\tilde{Y}_F+\tilde{Y}_O)
	+
	(\tilde{Y}_F+\tilde{Y}_O)
	\left(
		\frac{D  \tilde{\rho} }{D \tilde{t}}
		+
		\tilde{\rho} 
		\tilde{\nabla}\cdot\tilde{\mathbf{u}}
	\right)
        =
	\\
	\frac{(s_F+s_O)}{\tilde{Q}}
        \pmb{\tilde{\tau}}:\tilde{\nabla} \tilde{\mathbf{u}} 
        - 
	\frac{(s_F+s_O)}{\tilde{Q}}
        \tilde{\nabla} \cdot (\tilde{k}\tilde{\nabla} \tilde{T})
	\\
	\frac{(s_F+s_O)}{\tilde{Q}}
        \sum\limits_i 
        \tilde{\rho}
        \mathcal{D}_{im}
        \tilde{\nabla}
        \tilde{Y}_i     
        \cdot
        \left(
                c_{pi}
                \tilde{\nabla}
                \left(
                	\frac{\tilde{p}}{\tilde{\rho}\tilde{R}}
                \right)
        \right)
	-
	\frac{(s_F+s_O)}{\tilde{Q}}
        \sum\limits_i
	\tilde{h}_i
        s_i \omega	
	-
	\tilde{\nabla}\cdot( \tilde{\rho} \mathcal{D}_f\tilde{\nabla} (\tilde{Y}_F+\tilde{Y}_O))
\end{split}
\end{equation}

\begin{equation}
\begin{split}
        \frac{D }{D\tilde{t}}
	\left(
		\frac{(s_F+s_O)}{[\tilde{Q}(\tilde{\gamma}-1)]}
		\tilde{p}
		+
		\tilde{\rho} (\tilde{Y}_F+\tilde{Y}_O)
	\right)
	+
	\left(
		\frac{(s_F+s_O)}{[\tilde{Q}(\tilde{\gamma}-1)]}
		\tilde{\gamma}
		\tilde{p} 
		+
		\tilde{\rho} 
		(\tilde{Y}_F+\tilde{Y}_O)
	\right)
	\tilde{\nabla}\cdot\tilde{\mathbf{u}}
        =
	\\
	\frac{(s_F+s_O)}{\tilde{Q}}
        \pmb{\tilde{\tau}}:\tilde{\nabla} \tilde{\mathbf{u}} 
        - 
	\frac{(s_F+s_O)}{\tilde{Q}}
        \tilde{\nabla} \cdot (\tilde{k}\tilde{\nabla} \tilde{T})
	\\
	\frac{(s_F+s_O)}{\tilde{Q}}
        \sum\limits_i 
        \tilde{\rho}
        \mathcal{D}_{im}
        \tilde{\nabla}
        \tilde{Y}_i     
        \cdot
        \left(
                c_{pi}
                \tilde{\nabla}
                \left(
                	\frac{\tilde{p}}{\tilde{\rho}\tilde{R}}
                \right)
        \right)
	-
	\frac{(s_F+s_O)}{\tilde{Q}}
        \sum\limits_i
	\tilde{h}_i
        s_i \omega	
	-
	\tilde{\nabla}\cdot( \tilde{\rho} \mathcal{D}_f\tilde{\nabla} (\tilde{Y}_F+\tilde{Y}_O))
\end{split}
\end{equation}




















{\color{red}
Combining Eqs. (\ref{eqp}), (\ref{eqYO}) and (\ref{eqYF}) according to $(S+1)/(\gamma -1)Q$ Eq. (\ref{eqp}) + Eqs. (\ref{eqYO}) and (\ref{eqYF}):
%
\begin{equation}
	\tilde{\rho} \frac{D  }{D \tilde{t}}
	\left(
	\frac{(S+1)}{(\gamma-1)Q} \frac{ \tilde{p}}{ \tilde{\rho}} 
	+ \tilde{Y}_F + \tilde{Y}_O
	\right)
= 
	\tilde{\nabla}\cdot \left[ \tilde{\rho} \mathcal{D}
	\tilde{\nabla}
	\left(
	\frac{(S+1)}{(\gamma-1)Q} \frac{ \tilde{p}}{ \tilde{\rho}} 
	+ \tilde{Y}_F + \tilde{Y}_O
	\right)
	\right]
	+
	\frac{(S+1)}{(\gamma-1)Q} F
\end{equation}
Defining
\[
  H \equiv 
    \frac{1}{\tilde{Q}} 
    \frac{ \tilde{p}}{ \tilde{\rho}} + \tilde{Y}_F + \tilde{Y}_O,
    \ \ \ \ \ \ 
    \tilde{Q} \equiv 	\frac{(\gamma-1)Q}{(S+1)}
\]
\begin{equation}
	\tilde{\rho} \frac{\tilde{D}  H}{\tilde{D} \tilde{t}}
= 
	\tilde{\nabla}\cdot 
	\left( \tilde{\rho} \mathcal{D} \tilde{\nabla} H \right)
	+
	\frac{F}{\tilde{Q}}
\end{equation}
with (\ref{eqF})
\[
    F \equiv 
            (\gamma - 1)
        \left[
        \tilde{\nabla} \cdot 
		\left( 
		\tilde{\rho} \alpha
		\tilde{\nabla} (\frac{ \tilde{p} }{  \tilde{\rho} } )
		\right)
        -
        \tilde{p}  \nabla \cdot \mathbf{u}
		 +
        % \pmb{\tilde{\tau}}:\tilde{\nabla} \tilde{\mathbf{u}} 
        \pmb{\tilde{\tau}}:\tilde{\nabla} \pmb{\tilde{u}} 
        -
        \sum\limits_i
        \left(
                 \tilde{\rho}\tilde{h}_i
                \frac{D}{Dt}
		\tilde{Y}_i
		\right)
        +
        \tilde{\nabla} \cdot
        \sum\limits_i 
        \tilde{\rho}\mathcal{D}_{im}\tilde{\nabla}\tilde{Y}_i     
        \tilde{h}_i
        \right]
\]
}

{\color{blue}
In the fuel side of the flame, 
\[
     H
     =
     \frac{1}{\tilde{Q}} \frac{\tilde{p}}{\tilde{\rho}} + \tilde{Y}_F
     , \ \ \ 
     Z = S \tilde{Y}_F + 1
\]
and  combining them to determined $p$, it is found
\[
    H = \frac{1}{\tilde{Q}} \frac{\tilde{p}}{\tilde{\rho}}
    + 
    \frac{Z - 1}{S}, 
    \ \ \ \ \ \ 
    \frac{1}{\tilde{Q}} \frac{\tilde{p}}{\tilde{\rho}}
    =
    \left( H - \frac{Z-1}{S} \right)
\]
\begin{equation}
    \frac{\tilde{p}}{\tilde{\rho}}
    =
    \tilde{Q}
    \left( H - \frac{Z-1}{S} \right)
\label{eq0-15}
\end{equation}
and  in the oxidant side of the flame,
\[
     H
     =
     \frac{1}{\tilde{Q}} \frac{\tilde{p}}{\tilde{\rho}} + \tilde{Y}_O
     , \ \ \ 
     Z = - \tilde{Y}_O + 1
\]
and  combining them to determined $p$, it is found
\[
    H = \frac{1}{\tilde{Q}} \frac{\tilde{p}}{\tilde{\rho}}
    -
    (Z - 1), 
    \ \ \ \ \ \ 
    \frac{1}{\tilde{Q}} \frac{\tilde{p}}{\tilde{\rho}}
    =
    \left[ H + (Z-1) \right]
\]
\begin{equation}
    \frac{\tilde{p}}{\tilde{\rho}}
    =
    \tilde{Q}
    \left[ H + (Z-1) \right]
\label{eq0-16}
\end{equation}









From Eqs. (\ref{eq0-15}) and (\ref{eq0-16}), the source term $F$ can be written as a function of $H$ and $Z$, $F = F(H, Z, \rho, \mathbf{v}) $.

Note that $\hat{H}$ is a dimesionaless function because $p/q\rho$ is dimensionaless.
%
Then, rewriting function $H$ as
\[
    \hat{H} 
    \equiv 
    \frac{1}{\hat{Q} } \frac{\hat{p}}{\hat{\rho}} + \hat{Y}_F + \hat{Y}_O
\]
with
\[
    \hat{Q} 
    \equiv 
    \frac {(\gamma-1) q \rho_{\infty}} {(S+1)p_{\infty}}  
\]
Equation (38) in the dimensionaless form is
\begin{equation}
     St \frac{ \hat{D} }{ \hat{D} \hat{t}} \hat{H}
     = 
    \frac{1}{Re} \hat{\nabla} \cdot (  \hat{\rho} \hat{\alpha}  \hat{\nabla} \hat{H})  
      +
    \frac{\hat{F}}{\hat{Q}}  
\label{eq0-17}
\end{equation}
in which 
\[
    St \frac{ \hat{D} \bullet }{ \hat{D} \hat{t}} 
    \equiv 
    St \hat{\rho} \frac{\partial \bullet} {\partial \hat{t}}
      +
     \hat{\rho}  \pmb{\hat{u}} \cdot  ( \hat{\nabla}  \bullet)
    =
    St \hat{\rho} \frac{\partial \bullet } { \partial \hat{t} }
    +
    \hat{\rho} \hat{u}_i \frac{\partial \bullet} {\partial \hat{x}_i}  
\]
%\[    \hat{F}    \equiv    \frac{1}{\hat{Q}}    \frac{\gamma - 1}{\gamma}      \left[     -     \hat{p} \nabla \cdot \mathbf{\hat{v}}     +     \frac{1}{Re}     \nabla \cdot      \left(     \hat{\rho} \hat{\alpha}  \nabla ( \frac{\hat{p}}{\hat{\rho}} )     \right)      +     \frac{\gamma Ma^2}{Re}\nabla \mathbf{\hat{v}} : \mathbf{\hat{\Gamma}}      \right]\]
{\color{red}
\[
\hspace{-4cm}
    \hat{F} \equiv 
            (\gamma - 1)
        \left\{
        -
        \hat{p}  \hat{\nabla} \cdot \pmb{\hat{u}}
		 +
		 \frac{\gamma Ma^2}{Re}
        \pmb{\hat{\tau}}:\hat{\nabla} \pmb{\hat{u}}
        -
        \right.
\]
\[
        \left.
        \sum\limits_i
        \left(
                \frac{ \gamma St}
                     {\gamma-1}
                \hat{\rho} \hat{h}_i     
                \frac{\hat{D}}{\hat{D}\hat{t}}
		\hat{Y}_i
		\right)
        +
%        \\
        \hat{\nabla} \cdot
        \left[
        \sum\limits_i 
        \frac{ \gamma }{\gamma-1} 
        \frac{\hat{\rho}\mathcal{\hat{D}}_{im}}{Re}
        \hat{\nabla}\hat{Y}_i   
        \hat{h}_i
		+
		\frac{\hat{\rho} \hat{\alpha}}{Re}
		\hat{\nabla} (\frac{ \hat{p} }{ \hat{\rho} } )
		\right]
        \right\}
\]
}







The dimensionaless mixture fraction equation is similar the first three terms in Eq. (\ref{eq0-16})
\begin{equation}
     St\frac{\hat{D}}{\hat{D} \hat{t}} \hat{Z}
     = 
    \frac{1}{Re} \nabla \cdot (  \hat{\rho} \hat{\alpha}  \nabla \hat{Z})  
\label{eq0-18}
\end{equation}
with
%
\[
    \hat{t} = \frac{t}{t_c}, \ \ \ 
    \hat{x} \equiv \frac{x}{l_c}, \ \ \ 
\]
\[
    \pmb{\hat{u}} \equiv \frac{\pmb{u} }{ u_{\infty} }, \ \ \
    \hat{\rho}    \equiv \frac{\rho    }{ \rho_{\infty}}, \ \ \ 
    \hat{T}       \equiv \frac{T       }{ T_{\infty}}, \ \ \ 
    \hat{p}       \equiv \frac{p       }{ p_{\infty}}, \ \ \ 
    \hat{Y}_F     \equiv \frac{Y_F     }{ Y_{F\infty}}, \ \ \
    \hat{Y}_O     \equiv \frac{Y_O     }{ Y_{O-\infty}}, \ \ \ 
\]
\[
    \hat{\alpha} 
         \equiv \frac{\alpha           }
                     {\alpha_{\infty}  }, \ \ \ 
    Le_i \equiv \frac{ \alpha_{\infty} }
                     { D_{i\infty}     } , \ \ \ 
    Pe \equiv \frac{ l_c u_{\infty}    }
                   { \alpha_{\infty}   },  \ \ \ 
    Re \equiv \frac{ l_c u_{\infty}    } 
                   { \mu_{\infty} / \rho_{\infty} },  \ \ \ 
    Ma \equiv \frac{u_{\infty}^2}{\gamma R T_{\infty}}
\]
\[
    Pr=Le_F=Le_O=1,\ \ \ \ Pe = Re
\]
}

\newpage

\begin{equation}
\begin{split}
        \frac{D \tilde{p}}{D\tilde{t}}
	-
	\tilde{\gamma}
	\tilde{\nabla}\cdot{\mathbf{u}}
        =
	(\tilde{\gamma}-1)
        \pmb{\tilde{\tau}}:\tilde{\nabla} \tilde{\mathbf{u}} 
        + 
	(\tilde{\gamma}-1)
        \pmb{\tilde{\tau}}:\tilde{\nabla} \tilde{\mathbf{u}} 
        + 
	(\tilde{\gamma}-1)
        \tilde{\nabla} \cdot (\tilde{k}\tilde{\nabla} \tilde{T})
        +
	\\
	(\tilde{\gamma}-1)
        \sum\limits_i 
        \tilde{\rho}
        \mathcal{D}
        \tilde{\nabla}
        \tilde{Y}_i     
        \cdot
        \left(
                c_{pi}
                \tilde{\nabla}
                \left(
                	\frac{\tilde{p}}{\tilde{\rho}\tilde{R}}
                \right)
        \right)
	- 
	(\tilde{\gamma}-1)
        \sum\limits_i
        ({\color{red} h_i^0} + \tilde{h}_i)
        s_i	
        \omega
\end{split}
\end{equation}

Defining:

\begin{equation}
\tilde{\rho}\tilde{Q}w
\equiv
(\tilde{\gamma}-1)
\sum\limits_i
\tilde{h}_i
s_i	
\end{equation}

And mutliplied this equation by  $(s_o+s_f)/Q$

\begin{equation}
\begin{split}
	\frac{D }{D\tilde{t}}
	\left(
		\frac{\tilde{p}(s_o+s_f)}{Q}
	\right)
	+
	\tilde{\gamma}
	\frac{(s_o+s_f)}{Q}
	\tilde{\nabla}\cdot{\mathbf{u}}
        =
	(\tilde{\gamma}-1)
	\frac{(s_o+s_f)}{Q}
        \pmb{\tilde{\tau}}:\tilde{\nabla} \tilde{\mathbf{u}} 
        + 
	\\
	(\tilde{\gamma}-1)
	\frac{(s_o+s_f)}{Q}
        \pmb{\tilde{\tau}}:\tilde{\nabla} \tilde{\mathbf{u}} 
        + 
	(\tilde{\gamma}-1)
	\frac{(s_o+s_f)}{Q}
        \tilde{\nabla} \cdot (\tilde{k}\tilde{\nabla} \tilde{T})
        +
	\\
	(\tilde{\gamma}-1)
	\frac{(s_o+s_f)}{Q}
        \sum\limits_i 
        \tilde{\rho}
        \mathcal{D}
        \tilde{\nabla}
        \tilde{Y}_i     
        \cdot
        \left(
                c_{pi}
                \tilde{\nabla}
                \left(
                	\frac{\tilde{p}}{\tilde{\rho}\tilde{R}}
                \right)
        \right)
	- 
	(s_o+s_f)
	\tilde{\rho}w
\end{split}
\end{equation}

Adding it to the modified species equation, its getting:

\begin{equation}
\begin{split}
	\frac{D }{D\tilde{t}}
	\left(
		\frac{\tilde{p}(s_o+s_f)}{Q}
	\right)
	+
	\tilde{\rho} \frac{D  (\tilde{Y_o}+\tilde{Y_f})}{D \tilde{t}}
	+
	\tilde{\gamma}
	\frac{(s_o+s_f)}{Q}
	\tilde{\nabla}\cdot{\mathbf{u}}
        =
	(\tilde{\gamma}-1)
	\frac{(s_o+s_f)}{Q}
        \pmb{\tilde{\tau}}:\tilde{\nabla} \tilde{\mathbf{u}} 
        + 
	\\
	(\tilde{\gamma}-1)
	\frac{(s_o+s_f)}{Q}
        \pmb{\tilde{\tau}}:\tilde{\nabla} \tilde{\mathbf{u}} 
        + 
	(\tilde{\gamma}-1)
	\frac{(s_o+s_f)}{Q}
        \tilde{\nabla} \cdot (\tilde{k}\tilde{\nabla} \tilde{T})
        +
	\\
	(\tilde{\gamma}-1)
	\frac{(s_o+s_f)}{Q}
        \sum\limits_i 
        \tilde{\rho}
        \mathcal{D}
        \tilde{\nabla}
        \tilde{Y}_i     
        \cdot
        \left(
                c_{pi}
                \tilde{\nabla}
                \left(
                	\frac{\tilde{p}}{\tilde{\rho}\tilde{R}}
                \right)
        \right)
	+
	\tilde{\nabla}
	\cdot
	\left( 
		\tilde{\rho} 
		\left( 
			\mathcal{D}\tilde{\nabla} (\tilde{Y_o}+\tilde{Y_f})
		\right) 
	\right) 
\end{split}
\end{equation}

\begin{equation}
\begin{split}
	\frac{D }{D\tilde{t}}
	\left(
		\frac{\tilde{p}(s_o+s_f)}{Q}
	\right)
	+
	\tilde{\rho} \frac{\partial (\tilde{Y_o}+\tilde{Y_f})}{\partial \tilde{t}}
	+
	\tilde{\rho} {\mathbf{u}}\cdot\nabla( (\tilde{Y_o}+\tilde{Y_f})
	+
	\tilde{\gamma}
	\frac{(s_o+s_f)}{Q}
	\tilde{\nabla}\cdot{\mathbf{u}}
        =
	\\
	(\tilde{\gamma}-1)
	\frac{(s_o+s_f)}{Q}
        \pmb{\tilde{\tau}}:\tilde{\nabla} \tilde{\mathbf{u}} 
        + 
	\frac{(s_o+s_f)}{Q}
        \pmb{\tilde{\tau}}:\tilde{\nabla} \tilde{\mathbf{u}} 
        + 
	(\tilde{\gamma}-1)
	\frac{(s_o+s_f)}{Q}
        \tilde{\nabla} \cdot (\tilde{k}\tilde{\nabla} \tilde{T})
        +
	\\
	(\tilde{\gamma}-1)
	\frac{(s_o+s_f)}{Q}
        \sum\limits_i 
        \tilde{\rho}
        \mathcal{D}
        \tilde{\nabla}
        \tilde{Y}_i     
        \cdot
        \left(
                c_{pi}
                \tilde{\nabla}
                \left(
                	\frac{\tilde{p}}{\tilde{\rho}\tilde{R}}
                \right)
        \right)
	+
	\tilde{\nabla}
	\cdot
	\left( 
		\tilde{\rho} 
		\left( 
			\mathcal{D}\tilde{\nabla} (\tilde{Y_o}+\tilde{Y_f})
		\right) 
	\right) 
\end{split}
\end{equation}

\begin{equation}
\begin{split}
	\frac{D }{D\tilde{t}}
	\left(
		\frac{\tilde{p}(s_o+s_f)}{Q}
	\right)
	+
	\tilde{\rho} \frac{\partial (\tilde{Y_o}+\tilde{Y_f})}{\partial \tilde{t}}
	+
 	(\tilde{Y_o}+\tilde{Y_f})
	\left(
		\frac{\partial \tilde{\rho}}{\partial t}
		+
		\tilde{\nabla}\cdot(\rho\mathbf{u})
	\right)
	+
	\tilde{\rho} {\mathbf{u}}\cdot\nabla( (\tilde{Y_o}+\tilde{Y_f})
	+
	\tilde{\gamma}
	\frac{(s_o+s_f)}{Q}
	\tilde{\nabla}\cdot{\mathbf{u}}
        =
	\\
	(\tilde{\gamma}-1)
	\frac{(s_o+s_f)}{Q}
        \pmb{\tilde{\tau}}:\tilde{\nabla} \tilde{\mathbf{u}} 
        + 
	\frac{(s_o+s_f)}{Q}
        \pmb{\tilde{\tau}}:\tilde{\nabla} \tilde{\mathbf{u}} 
        + 
	(\tilde{\gamma}-1)
	\frac{(s_o+s_f)}{Q}
        \tilde{\nabla} \cdot (\tilde{k}\tilde{\nabla} \tilde{T})
        +
	\\
	(\tilde{\gamma}-1)
	\frac{(s_o+s_f)}{Q}
        \sum\limits_i 
        \tilde{\rho}
        \mathcal{D}
        \tilde{\nabla}
        \tilde{Y}_i     
        \cdot
        \left(
                c_{pi}
                \tilde{\nabla}
                \left(
                	\frac{\tilde{p}}{\tilde{\rho}\tilde{R}}
                \right)
        \right)
	+
	\tilde{\nabla}
	\cdot
	\left( 
		\tilde{\rho} 
		\left( 
			\mathcal{D}\tilde{\nabla} (\tilde{Y_o}+\tilde{Y_f})
		\right) 
	\right) 
\end{split}
\end{equation}



\end{document}
